\chapter{Законы сохранения}
Продолжим развитие теории предельных переходов, потому что нам
эти знания нужны для выводов законов сохранения.
Рассмотрим более сложный случай, когда функция зависит от нескольких
переменных, например $f=f(x,y,z).$ 
Также как и ранее будем рассмотривать только те функции, которые можно представить
в виде ряда $f(x,y,z) = \sum\limits_{i}\sum\limits_{j}
\sum\limits_{k}c_{ijk} (x-x_0)^i(y-y_0)^j(z-z_0)^k,$ при этом
ряд может быть как конечным, так и бесконченым.
Этот ряд можно переписать в виде $f(x,y,z) = c_{000}
+c_{100}(x-x_0) + c_{010} (y-y_0) + c_{001} (z-z_0) + 
+c_{200}(x-x_0)^2 + c_{020} (y-y_0)^2 + c_{002} (z-z_0)^2 + 
+c_{110}(x-x_0)(y-y_0) + c_{011} (y-y_0)(z-z_0) + c_{101} (z-z_0)(x-x_0) + \dots.$
Как и ранее введём обозначения 
$\Delta x = x-x_0,$
$\Delta y = y-y_0,$
$\Delta z = z-z_0.$
Ясно, что при $x=x_0,$ $y=y_0,$ $z=z_0$ $f(x,y,z)=f(x_0,y_0,z_0)$ и 
$f(x,y,z) = c_{000},$ откуда $c_{000} = f(x_0,y_0,z_0).$
Тогда 
\begin{equation}
\label{eq:full_add}
\begin{split}
\Delta f 
=&c_{100}(x-x_0) + c_{010} (y-y_0) + c_{001} (z-z_0)  +\\
+&c_{200}(x-x_0)^2 + c_{020} (y-y_0)^2 + c_{002} (z-z_0)^2  +\\
+&c_{110}(x-x_0)(y-y_0)+c_{011} (y-y_0)(z-z_0)+\\
+&c_{101} (z-z_0)(x-x_0) + \dots  %=\\
%=&c_{100}(x-x_0) + c_{010} (y-y_0) + c_{001} (z-z_0)  +\\
%+&\alpha \Delta x + \beta \Delta y + \gamma \Delta z,
\end{split}
\end{equation}
%при этом
%$\alpha, \beta, \gamma \to 0$ при $\Delta x, \Delta y, \Delta z \to 0.$
Будем понимать под выражением $\lim\limits_{\Delta \xi \to 0}$ случай, когда
$\xi \to 0,$ и при этом остальные переменные не варьируются. Имеем тогда
$\lim\limits_{\Delta x \to 0} \frac{\Delta f(x,y,z)}{\Delta x} = c_{100},$
$\lim\limits_{\Delta y \to 0} \frac{\Delta f(x,y,z)}{\Delta y} = c_{010},$
$\lim\limits_{\Delta z \to 0} \frac{\Delta f(x,y,z)}{\Delta z} = c_{001}.$
Положим теперь, что  $x = x(t),$ $y=y(t),$ $z=z(t).$ Нас интересует
выражение 
$\lim\limits_{\Delta t \to 0} \frac{\Delta f(x,y,z)}{\Delta t}.$
Представим функции $x(t), y(t), z(t)$ в виде рядов:
$x(t) = x(t_0) + \lim \limits_{\Delta t \to 0} \frac {\Delta x (t)} {\Delta t} \Delta t + \theta_x \Delta^2 t ,$
$y(t) = y(t_0) + \lim \limits_{\Delta t \to 0} \frac {\Delta y (t)} {\Delta t} \Delta t + \theta_y \Delta^2 t ,$
$x(t) = z(t_0) + \lim \limits_{\Delta t \to 0} \frac {\Delta z (t)} {\Delta t} \Delta t + \theta_z \Delta^2 t ,$
$\theta_x, \theta_y, \theta_z$ --- некие функции $t_0.$
Подставим $x, y, z$ 
 в таком виде в 
$\lim\limits_{\Delta t \to 0} \frac{\Delta f(x,y,z)}{\Delta t},$
получим из (\ref{eq:full_add})
\begin{equation}
\label{eq:complex_diff}
\begin{split}
\lim\limits_{\Delta t \to 0} \frac{\Delta f(x,y,z)}{\Delta t} &=\\ 
&\lim\limits_{\Delta x \to 0} \frac{\Delta f(x,y,z)}{\Delta x} 
\lim\limits_{\Delta t \to 0} \frac{\Delta x}{\Delta t}+ \\
&\lim\limits_{\Delta y \to 0} \frac{\Delta f(x,y,z)}{\Delta y} 
\lim\limits_{\Delta t \to 0} \frac{\Delta y}{\Delta t}+ \\
&\lim\limits_{\Delta z \to 0} \frac{\Delta f(x,y,z)}{\Delta z} 
\lim\limits_{\Delta t \to 0} \frac{\Delta z}{\Delta t}.
\end{split}
\end{equation}
\section{Закон сохранения энергии и импульса}
Вообще говоря уравнения движения и законы сохранения выводятся из более общего принципа
наименьшего действия. Однако мы за постулат возьмём закон сохранения энергии.\par
Пусть у нас есть система (различных) материальных точек (частиц) и пусть эта система не взаиможействует
с внешними материальными точками (внешних частиц либо нет, либо их воздействием можно
пренебречь). Введём некую векторную динамическую характеристику каждой частицы $\vec{p}$ (импульс).
В случае, если внутри системы материальные точки не взаимодействуют, утверждается, что
сохраняется величина $E = \sum\limits_{i} T_i = \sum\limits_{i} \alpha_i \vec{p_i}^2,$
которая называется полной энергией системы. $T_i = \alpha_i \vec{p_i}^2$ --- кинетическая
энергия $i$-й материальной точки, $\alpha_i$ --- некоторый коэффициент, характеризующий
$i$-ю материальную точку. В случае, если частицы системы взаимодействуют друг с другом,
часто можно ввести понятия энергии взаимодействия $U=U(\vec{r_1},\vec{r_2},\dots,\vec{r_n})$
как добавку к суммарной кинетичской энергии; постулируется, что сохраняется
величина $E = \sum\limits_{i} T_i + U(\vec{r_1},\vec{r_2},\dots,\vec{r_n}),$ которая в данном случае
называется полной энергией.\par
Снова вернёмся к случаю невзаимодействующих частиц.
Так как $E = const,$ то $\lim\limits_{\Delta t \to 0} \frac{\Delta E}{\Delta t} = 0.$
Найдём выражение для $\lim\limits_{\Delta t \to 0} \frac{\Delta \vec{p_i}}{\Delta t}.$
Воспользуемся выражением (\ref{eq:complex_diff}), получим
$\lim\limits_{\Delta t \to 0} \frac{\Delta \vec{p_i}(x,y,z)}{\Delta t} =
\lim\limits_{\Delta x \to 0} \frac{\Delta \vec{p_i}(x,y,z)}{\Delta x} 
\lim\limits_{\Delta t \to 0} \frac{\Delta x}{\Delta t}+ \\
\lim\limits_{\Delta y \to 0} \frac{\Delta \vec{p_i}(x,y,z)}{\Delta y} 
\lim\limits_{\Delta t \to 0} \frac{\Delta y}{\Delta t}+ \\
\lim\limits_{\Delta z \to 0} \frac{\Delta \vec{p_i}(x,y,z)}{\Delta z} 
\lim\limits_{\Delta t \to 0} \frac{\Delta z}{\Delta t}.$
Мы знаем, что
$\vec{v} = \lim\limits_{\Delta t \to 0} \frac{\Delta \vec{r}}{\Delta t},$
или покомпонентно
${v_x} = \lim\limits_{\Delta t \to 0} \frac{\Delta x}{\Delta t},$
${v_y} = \lim\limits_{\Delta t \to 0} \frac{\Delta y}{\Delta t},$
${v_z} = \lim\limits_{\Delta t \to 0} \frac{\Delta z}{\Delta t},$ 
