\chapter{Статика}
\section{Векторное произведение}
Рассмотрим трёхмерное пространство с ортонормированным базисом $\{\vec{i}, \vec{j}, \vec{k}\}.$
Разложим произвольные вектора пространства $\vec{a},$ $\vec{b}$ по этому базису, тогда
$\vec{a} = a_x \vec{i} + a_y \vec{j} + a_z \vec{k},$
$\vec{b} = b_x \vec{i} + b_y \vec{j} + b_z \vec{k}.$
\begin{definition}[Векторное произведение]
$$[\vec{a} \times \vec{b}] = \begin{pmatrix}
	a_y b_z - a_z b_y \\
	a_z b_x - a_x b_z \\
	a_x b_y - a_y b_x \end{pmatrix}
$$\end{definition}
\begin{task}
Покажите корректность определения векторного произведения
(результат не зависит от выбора базиса)\end{task}
Также векторное произведение $\vec{a} \times \vec{b}$ определяется
как вектор, ортогональный плоскости, посторенной на векторах $\vec{a},$
$\vec{b},$ норма которого равна $\left\Vert \vec{a} \right \Vert 
\left \Vert \vec{b} \right \Vert \left \vert \sin \varphi \right \vert,$
$\varphi$ --- угол между векторами $\vec{a}$ и $\vec{b}.$ Направление
выбирается по правилу буравчика.
Свойства векторного произведения:
\begin{enumerate}
\item $[\vec{a} \times \vec{b}] = - [\vec{b} \times \vec{a}]$
\item $[[\vec{a} \times \vec{b}] \times \vec{c} ] + [[\vec{b} \times \vec{c}] \times \vec{a}]
+ [[\vec{c} \times \vec{a}]\times\vec{b}] = 0$
\item Линейность
\item $[\vec{a} \times [\vec{b} \times \vec{c}]] = 
\vec{b} \langle \vec{a} \vert \vec{c}\rangle-
\vec{c} \langle \vec{a} \vert \vec{b}\rangle$
\end{enumerate}
\section{Момент импульса}
В случае отсутствия внешнего воздействия система,
кроме перечисленных свойств (однородность пространства и времени),
обладает также свойством изотропии пространства, а именно
все направления в пространстве равнозначны. Исходя из
данного постулата, а также
из уже полученных уравнений движения (законов Ньютона),
мы можем найти такую величину (величины),
которые бы не изменялись с течением времени в процессе движения системы.
Выберем некое направление в пространстве, и положим ось $OZ$ вдоль этого
направления (соответствует базисному вектору $\vec{k}$). Две других
оси расположим взаимноортогонально с $OZ,$ соответствующие
базисные вектора --- $\vec{i}$ (ось $OX$) и $\vec{j}$ (ось $OY$).
Разложим вектор импульса $\vec{p}$ и вектор координат $\vec{r}$ на компоненты
в этом базисе, 
$\vec{p} = p_x \vec{i} + p_y \vec{j} + p_z \vec{k},$
$\vec{r} = x \vec{i} + y \vec{j} + z \vec{k}.$
Рассмотрим комбинацию $L_z = x p_y - y p_x.$
Рассмотрим значение этой же комбинации в другом базисе, полученном
поворотом вокруг $\vec{k}$ на угол $\varphi$:
\begin{align*}
\vec{i'} &= \vec{i} \cos \varphi + \vec{j} \sin \varphi\\
\vec{j'} &= - \vec{i} \sin \varphi + \vec{j} \cos \varphi.
\end{align*}
