\chapter{Динамика}
Рассмотрим закон сохранения энергии для одномерного движения в виде
\begin{equation}
\alpha p^2 + U (x) = {const},
\label{eq:cons_law_energy}
\end{equation}
где $U(x)$ --- потенциальная энергия, $\alpha \in \mathbb{R}_+.$
Так как импульс $p$ является функцией времени $t,$ координата $x$
также является функцией времени, можно представить $p$ как функцию координаты
$p = p(x(t)).$
Проделаем следующую операцию над выражением (\ref{eq:cons_law_energy}):
$$\lim\limits_{\Delta x \to 0} \frac{\Delta\left(\text{выражение \ref{eq:cons_law_energy}}\right)}{\Delta x}.$$
Подставляя выражение получим
$$\lim\limits_{\Delta x \to 0} \frac{\Delta\alpha p^2}{\Delta x}+
\lim\limits_{\Delta x \to 0} \frac{\Delta U(x)}{\Delta x}=0.$$
Вспомним определение силы $F = -\lim\limits_{\Delta x \to 0}\Delta U(x) / \Delta x.$
Для вычисления такого предельного перехода от кинетической энергии $\alpha p^2$ 
проделаем ряд действий.
Нетрудно убедится, что
$\lim\limits_{\Delta x \to 0} \Delta\alpha p^2/{\Delta x}=$
$\lim\limits_{\Delta x \to 0} \alpha\Delta p^2/{\Delta x}.$
Поделим и домножим подпредельное выражение на $\Delta p \to 0.$
Сам предел представим в виде произведения двух пределов.
Имеем
$$\lim\limits_{\Delta x \to 0} \alpha\frac{\Delta p^2}{\Delta x}=
\lim\limits_{\Delta p \to 0} \alpha\frac{\Delta p^2}{\Delta p}
\lim\limits_{\Delta x \to 0} \frac{\Delta p}{\Delta x}.$$
Выражение, аналогичное $\lim\limits_{\Delta p \to 0} \alpha\frac{\Delta p^2}{\Delta p}$
мы уже вычисляли в главе ???кинематика. Результатом вычислений стало выражение
$2 p.$
Теперь проделаем такую же процедуру, домножив и поделив на $\Delta t \to 0.$
По определению скорости $v = \lim\limits_{\Delta t \to 0} \Delta x / \Delta t,$
откуда окончательно имеем следующее равенство
$$\frac{2\alpha p}{v} \lim \limits_{\Delta t \to 0} \frac{\Delta p} {\Delta t} = F.$$
В частности, если определить импульс как $2 \alpha v,$ то выражение упростится до
$$\lim\limits_{\Delta t \to 0} \frac{\Delta p} {\Delta t} = F.$$
Принято коэффициент $2\alpha$ называть массой $m.$
В трёхмерном случае проделанная процедура выполняется для каждой компоненты координат независимо
друг от друга. По определению сила есть вектор $\vec{F} =
-\vec{i}\lim\limits_{\Delta x \to 0}
\frac{\Delta U (x,y,z)}{\Delta x}-$ 
$\vec{j}\lim\limits_{\Delta y \to 0}
\frac{\Delta U (x,y,z)}{\Delta y}-$ 
$\vec{k}\lim\limits_{\Delta z \to 0}
\frac{\Delta U (x,y,z)}{\Delta z}.$
Имеем
\begin{equation}
\lim\limits_{\Delta t \to 0}\frac{\Delta \vec{p}}{\Delta t} = \vec{F}.
\label{eq:2nd_Newton}
\end{equation}
Это выражение очень пригодится нам в дальнейшем.\par
Рассмотрим взаимодействие между двумя материальными точками.
По закону сохранения импульса $\vec{p}_1 + \vec{p}_2 = \overrightarrow{const}.$
Подействуем предельным переходом $\lim\limits_{\Delta t \to 0} \frac{\Delta}{\Delta t}$
на это выражение. Получим
$$\lim\limits_
{\Delta t \to 0}\frac
{\Delta \vec{p}_1}
{\Delta t}
+ \lim\limits_
{\Delta t \to 0} \frac
{\Delta \vec{p}_2}
{\Delta t}
= 0.$$
Подставим в это выражение определение силы, получим
\begin{equation}
\vec{F}_{21} = -\vec{F}_{12}
\label{eq:3rd_Newton}
\end{equation}
где введены обозначения $\vec{F}_{21}$ --- сила, действующая
со стороны второй материальной точки на первую, $\vec{F}_{12}$ --- сила,
действующая со стороны первой материальной точки на вторую.
\begin{axiom}[Принцип суперпозиции]
Действие нескольких материальных точек на одну материальную точку можно
эквивалентно представить действием равнодействующей силы $\vec{R} = \sum\limits_{i}\vec{F}_i.$
\end{axiom}
\section{Законы Ньютона}
\subsection{Динамика материальной точки}
Будем рассматривать материальные точки с неизменной массой $m = 2\alpha.$
\begin{theorem}[Первый закон Ньютона]
Если нет сил, действующих на материальную точку, то она движется равномерно и прямолинейно.
С такой точкой можно связать систему отсчёта, которая называется инерциальной системой
отсчёта (ИСО). Все инерциальные системы отсчёта равноправны (то есть законы механики
в различных ИСО одинаковы), при переходе от одной системы
отсчёта к другой выполняются преобразования Галилея.
\begin{proof}
В силу (\ref{eq:2nd_Newton}) имеем при $\vec{F} = 0$ $\vec{p} = \overrightarrow{const},$
по определению импульса $\vec {p} = m \vec{v},$ откуда при неизменной массе $\vec{v}=\overrightarrow{const}.$
Остальная часть закона --- принцип относительности Галилея --- была пояснена в
разделе кинематика???.
\end{proof}
\end{theorem}
\begin{theorem}[Второй закон Ньютона]
\begin{equation}
\sum \limits_{i=1}^{n}\vec{F}_i = m \vec{a},
\label{eq:2nd_Newton_point}
\end{equation}
где $\vec{F}_i$ --- сила со стороны $i$-го материальной точки
из системы $n$ точек на материальную точку, $m$ --- масса материальной точки,
$\vec{a}$ --- ускорение материальной точки.
\begin{proof}
Применим выражение (\ref{eq:2nd_Newton}), принцип суперпозиции и предположение
о неизменности массы материальной точки.
\end{proof}
\end{theorem}
\begin{theorem}[Третий закон Ньютона]
При взаимодействии двух материальных точек
\begin{equation}
\vec{F}_{12}=-\vec{F}_{21},
\label{eq:3rd_Newton_point}
\end{equation}
где $\vec{F}_{21}$ --- сила, действующая
со стороны второй материальной точки на первую, $\vec{F}_{12}$ --- сила,
действующая со стороны первой материальной точки на вторую.
\begin{proof}
Эквивалентно доказательству (\ref{eq:3rd_Newton})
\end{proof}
\end{theorem}
Приведённые выше законы позволяют решать механические задачи для
любой системы материальных точек, если известны силы между этими  точками,
а так же известны начальные условия --- координаты и импульсы (или скорости)
в начальный момент времени.
\subsection{Движение системы материальных точек}
Достаточно часто система материальных точек является очень сложной, либо
у нас отсутствует знание о природе сил взаимодействия между точками системы
или знания о начальных параметрах системы. Однако в некоторых случаях
удаётся решить механическую задачу для таких систем, например, в случае
твёрдого тела, когда материальные точки не меняют своего расположения относительно друг друга.
Такой случай достаточно часто встречается в задачах, иногда
такие задачи усложняются тем, что тела меняют массу (при этом можно, конечно, полагать,
что материальные точки, составляющие тело, массу свою не меняют, а лишь покидают тело).
Оказывается, в случае системы материальных точек существует точка в пространстве такая, что
для неё применимы законы Ньютона. Рассмотрим этот факт более детально.
Выделим из всех $N$ материальных точек $n$ точек и объединим их (<<мысленно>>) в систему материальных точек.
\begin{definition}[Центр масс]
$$\vec{r}_\text{Ц.М.} = \frac{\sum\limits_{i=1}^{n}m_i \vec{r}_i}{\sum\limits_{i=1}^{n} m_i}.$$
\label{def:cm}
\end{definition}
По определению скорость центра масс получается
$$\vec{v}_\text{Ц.М.} = \frac{\sum\limits_{i=1}^{n}m_i \vec{v}_i}{\sum\limits_{i=1}^{n} m_i}
= \frac{\sum\limits_{i=1}^{n} \vec{p}_i}{M},$$
где $M$ --- суммарная масса, откуда следует, что
$$\vec{p}_\text{Ц.М.} = M \vec{v}_\text{Ц.М.}
= {\sum\limits_{i=1}^{n} \vec{p}_i}.$$
\begin{theorem}[Второй закон Ньютона для системы материальных точек]
$$\sum\limits_{i=1}^{N-n}\vec{F}_\text{внеш.} = \frac{\Delta \vec{p}_\text{Ц.М.}}{\Delta t}.$$
\label{th:2nd_Newton_system}
\begin{proof}
Основывается на законе сохранения импульса и принципе суперпозиции.
\end{proof}
\end{theorem}
Также для системы материальных точек верен третий закон Ньютона $\vec{F}_{12} = - \vec{F}_{21}.$
\section{Силы в природе}
Рассмотрим силы, которые часто встречаются в задачах.
\begin{definition}[Гравитация]
Между двумя материальными точками, обладающими массами, возникает сила
притяжения, направленная по линии, соединяющей эти точки.
$$\vec{F}_\text{gr12} = -G \frac{m_1m_2}{r_{12}^3}\vec{r_{12}},$$
где индексы 1,2 обозначают <<от первого на/до второго>>,
$G$ --- гравитационная постоянная.
\label{def:grav}
\end{definition}
В случае системы материальных точек такие силы возникают между всеми парами точек,
при этом задача нахождения уравнений движений заметно усложняется (например, если мы
рассмотрим три материальные точки, взаимодействующие гравитационно, то такая задача
не решается в общем виде аналитически). В случае взаимодействия тел, состоящих из
бесконечного множества точек при наличии некоторой симметрии задача о взаимодействии может быть
решена аналитически методами векторного анализа. В частности, точное решение
для взаимодействующих массивных шаров с равномернораспределённой плотностью совпадает
с формулировкой закона \ref{def:grav}, где $r_{12}$ --- расстояние между центрами сфер (шаров).
Если будем рассматривать взаимодействие массивного шара с материальной точкой с многим меньшей массой,
то оказывается, что вблизи поверхности сила взаимодействия между телами почти не зависит от
расстояния между ними. Такую силу называют силой тяжести.
\begin{definition}[Сила тяжести]
$$\vec{F} = m\vec{g},$$
где $\vec{g}$ --- вектор напряжённости гравитационного поля
или, как его ещё называют, ускорение свободного падения.
\end{definition}
\begin{task}
Выразить $g$ через $G,$ $R$ (радиус Земли) и $M$ (масса Земли).
\end{task}
\begin{definition}[Сила натяжения нити]
В случае невесомой нерастяжимой нити в точке крепления
данной нити к телу возникает сила $\vec{T},$ направленная вдоль нити, приложенная к телу.
Если мысленно вырезать из нити небольшой кусок, то в силу его безмассовости
сумма сил, действующих на него, равна нулю. Поэтому в каждой точке нити модуль
силы натяжения равен $T.$ Величина $T$ вообще говоря может быть произвольной,
больше нуля и ограниченной сверху прочностью нити (иногда встречаются такие задачи).
\end{definition}
\begin{definition}[Сила реакции опоры]
Сила реакции опоры $\vec{N}$ действует при соприкосновении двух любых твёрдых
тел и направлена перпендикулярно плоскости касания так, чтобы тела расталкивались.
\end{definition}
\begin{definition}[Вес тела]
Вес тела --- частный случай силы реакции опоры.
Вес -- это сила, с которой тело действует на опору или подвес.
В задачах термин вес встречается очень редко, нахождение веса обычно сводится
к нахождению силы реакции опоры со стороны <<опоры>> на тело, а затем применяется третий закон Ньютона.
\end{definition}
\begin{definition}[Сила трения]
Очень странная сила. В задачах встречается два типа данной силы, и необходимо
хорошо научится различать эти случаи.\par
\textbf{Сила трения покоя} --- сила, направленная параллельно поверхности взаимодействия так,
чтобы суммарная сила, действующая на тело, была равна нулю. Иная формулировка направления силы:
против предполагаемого движения. Величина силы трения покоя ограничена сверху,
предельное значение связано с нормой силы реакции опоры $N;$ $F_\text{тр.} = \mu N.$
\par\textbf{Сила трения скольжения} --- сила, направленная против относительной скорости
тела и поверхности, приложенная к телу параллельно поверхности соприкосновения. Величина
её вообще говоря зависит от многих параметров, однако в школьной физике для просты
кладут её $\mu N,$ коэффициент трения $\mu$ обычно меньше единицы. Это приближение
достаточно хорошо работает на практике, так, нести тяжёлые вещи проще волоком,
нежели на весу (а ещё лучше приделать к телу колёсики с подшипниками и смазать их машинным маслом,
что уменьшает трение и упрощает перенос всяких чемоданов).
\end{definition}
\begin{definition}[Сила упругости пружины]
$$\vec{F} = - k \Delta \vec{x},$$
k --- положительный коэффициент упругости, $\Delta \vec{x}$ --- удлинение/укорочение пружины.
\end{definition}
\begin{definition}[Сила джедаев]
Энергетическое поле, окружающее все материальные точки
давным давно, в далёкой-далёкой галактике. В наше время экспериментально не
наблюдается, поэтому авторы задач редко используют данную силу в своём творчестве.
\end{definition}

\section{Решение задач динамики}
Приведём алгоритм решения динамических задач.
\begin{enumerate}
\item Сделать чертёж.
\item Выбрать систему материальных точек (тел), для которой будет записываться второй  закон Ньютона
(этот выбор, вообще говоря, произволен, что открывает простор для составления олимпиадных задач,
которые могут решаться только при удачном выборе системы тел).
\item Выбрать момент времени, для которого будет записываться второй закон Ньютона.
При равноускоренном движении, когда силы не зависят от времени, можно рассматривать любой
момент времени, а затем обобщить результат на весь временной интервал. Вообще говоря, данный пункт
может выполнятся раньше предыдущего пункта. В некоторых сложных задачах
удачный выбор момента времени также упрощает решение задачи.
\item Сделать чертёж для выбранной системы тел (материальных точек) в выбранный момент времени.
\item Отметить на чертеже действующие на выбранную систему тел в выбранный момент времени
внешние силы. Придерживайтесь правила <<одно стороннее тело (тело или материальная точка, не входящие в
выбранную систему тел) --- как минимум одна внешняя сила>>.
\item Указать на чертеже направление ускорения (обычно это направление достаточно просто находится).
\item Выбрать инерциальную систему отсчета.
Ввести систему координат, выбрать начало отсчёта. Вообще говоря, выбор координат и инерциальной
системе отсчёта в динамике может быть
произведен произвольным образом, что открывает также простор для составителей олимпиадных задач.
\item Решить геометрические подзадачи данной задачи. Часто бывает необходимо найти углы между силами
и направлением ускорения или выбранной системой координат. Понятно, что систему координат
желательно выбрать так, чтобы как можно большее количество геометрических подзадач стало тривиальным.
В частности, если большая часть сил расположено в плоскости и перпендикулярны друг другу, следует
выбрать оси сонаправленными с этими силами.
\item Записать в векторном виде второй закон Ньютона.
\item Сделать проекции сил и ускорения тела на оси. Записать
второй закон Ньютона в проекциях (для двумерной задачи получается два уравнения).
\item Найти условия прекращения/начала/etc движения. Обычно они формулируются в тексте задачи.
\item Найти кинематические связи в задаче. Как ни странно, это основная сложность динамических
задач. Общего рецепта для составления уравнений связей нет, некоторые случаи были разобраны
в разделе кинематика???.
\item Составить кинематические уравнения.
\item Выписать получившиеся уравнения. Обычно для решения задачи достаточно
равенства числа уравнений и числа неизвестных, однако это не так. В случае нехватки
уравнений возможно потребуется рассмотреть другую систему тел или другой момент времени.
Также часто бывает, что число уравнений на одно меньше, но при этом ответ на вопрос в задаче может
быть получен. Таким образом физическая задача о движении тела под действием
сил сводится к системе математических уравнений. К счастью, составители задач стараются
формулировать условия так, чтобы система математических уравнений решалась школьными методами.
В случае невозможности решить систему возможно следует поискать либо ошибку в решении, либо условие,
при котором некоторыми членами уравнения можно пренебречь и упростить себе математическую задачу.
Этот навык -- вершина мастерства физика, решающего задачу. Умение правильно и обоснованно пренебрегать
приобретается с опытом и глубоким пониманием математических основ приближённых методов. Обычно
в школьных задачах такое умение не требуется.
\end{enumerate}

